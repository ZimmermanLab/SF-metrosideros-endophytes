%% Submissions for peer-review must enable line-numbering
%% using the lineno option in the \documentclass command.
%%
%% Preprints and camera-ready submissions do not need
%% line numbers, and should have this option removed.
%%
%% Please note that the line numbering option requires
%% version 1.1 or newer of the wlpeerj.cls file, and
%% the corresponding author info requires v1.2

\documentclass[fleqn,10pt,lineno]{wlpeerj} % for journal submissions

% ZNK -- Adding headers for pandoc

\setlength{\emergencystretch}{3em}
\providecommand{\tightlist}{
\setlength{\itemsep}{0pt}\setlength{\parskip}{0pt}}
\usepackage[unicode=true]{hyperref}
\usepackage{longtable}


% Pandoc syntax highlighting
% See https://github.com/rstudio/rticles/issues/182

% Pandoc citation processing
\newlength{\csllabelwidth}
\setlength{\csllabelwidth}{3em}
\newlength{\cslhangindent}
\setlength{\cslhangindent}{1.5em}
% for Pandoc 2.8 to 2.10.1
\newenvironment{cslreferences}%
  {}%
  {\par}
% For Pandoc 2.11+
\newenvironment{CSLReferences}[2] % #1 hanging-ident, #2 entry spacing
 {% don't indent paragraphs
  \setlength{\parindent}{0pt}
  % turn on hanging indent if param 1 is 1
  \ifodd #1 \everypar{\setlength{\hangindent}{\cslhangindent}}\ignorespaces\fi
  % set entry spacing
  \ifnum #2 > 0
  \setlength{\parskip}{#2\baselineskip}
  \fi
 }%
 {}
\usepackage{calc} % for calculating minipage widths
\newcommand{\CSLBlock}[1]{#1\hfill\break}
\newcommand{\CSLLeftMargin}[1]{\parbox[t]{\csllabelwidth}{#1}}
\newcommand{\CSLRightInline}[1]{\parbox[t]{\linewidth - \csllabelwidth}{#1}\break}
\newcommand{\CSLIndent}[1]{\hspace{\cslhangindent}#1}

% Pandoc Header
\usepackage[tablesfirst, heads, nolists, nomarkers]{endfloat}
\usepackage{booktabs}
\usepackage{longtable}
\usepackage{array}
\usepackage{multirow}
\usepackage{wrapfig}
\usepackage{float}
\usepackage{colortbl}
\usepackage{pdflscape}
\usepackage{tabu}
\usepackage{threeparttable}
\usepackage{threeparttablex}
\usepackage[normalem]{ulem}
\usepackage{makecell}
\usepackage{xcolor}

\title{Urban biogeography of fungal endophytes across San Francisco}

\author[1]{Emma Gibson}

\author[2]{Naupaka Zimmerman}

\corrauthor[2]{Naupaka Zimmerman}{\href{mailto:nzimmerman@usfca.edu}{\nolinkurl{nzimmerman@usfca.edu}}}

\affil[1]{Department of Biology, University of San Francisco}
\affil[2]{Department of Biology, University of San Francisco}


%
% \author[1]{First Author}
% \author[2]{Second Author}
% \affil[1]{Address of first author}
% \affil[2]{Address of second author}
% \corrauthor[1]{First Author}{f.author@email.com}

% 

\begin{abstract}
In natural and agricultural systems, the plant microbiome--the microbial organisms associated with plant tissues--has been shown to have important effects on host physiology and ecology, yet we know little about how these plant-microbe relationships play out in urban environments. Here we characterize the composition of fungal communities associated with leaves of one of the most common sidewalk trees in the city of San Francisco, California. We focus our efforts on endophytic fungi (asymptomatic microfungi that live inside healthy leaves), which have been shown in other systems to have large ecological effects on the health of their plant hosts. Specifically, we characterized the foliar fungal microbiome of \emph{Metrosideros excelsa} trees growing in a variety of urban environmental conditions. We used high-throughput culturing, PCR, and Sanger sequencing of the ITS nrDNA region to quantify the composition and structure of fungal communities growing within healthy leaves of 30 \emph{M. excelsa} trees from 6 distinct sites, which were selected to capture the range of environmental conditions found within city limits. Sequencing resulted in 921 high-quality ITS sequences. These sequences clustered into 88 Operational Taxonomic Units (97\% VSEARCH OTUs). We found that these communities encompass relatively high alpha (within) and beta (between-site) diversity. Because the communities are all from the same host tree species, and located in relatively close geographical proximity to one another, these analyses suggest that urban environmental factors such as urban heat islands or differences in traffic density (and associated air quality) could potentially be influencing the composition of these fungal communities. These biogeographic patterns provide evidence that plant microbiomes in urban environments can be as dynamic and complex as their natural counterparts. As human populations continue to transition out of rural areas and into cities, understanding the factors that shape environmental microbial communities in urban ecosystems stands to become increasingly important.
% Dummy abstract text. Dummy abstract text. Dummy abstract text. Dummy abstract text. Dummy abstract text. Dummy abstract text. Dummy abstract text. Dummy abstract text. Dummy abstract text. Dummy abstract text. Dummy abstract text.
\end{abstract}

\begin{document}

\flushbottom
\maketitle
\thispagestyle{empty}

\hypertarget{introduction}{%
\section*{Introduction}\label{introduction}}
\addcontentsline{toc}{section}{Introduction}

Although major cities and urban centers only cover a small portion the Earth's total geographic area, more than 50\% of the human population lives in these urban centers, and the impact that these cities have on the environment can be seen worldwide (Schneider, Friedl, and Potere 2009). As people continue to move to urban environments, understanding the ecology of cities and urban settings will become critical to human health and well being. Despite their comparatively small geographic size, the high density of human populations in these environments makes them distinct ecosystems with their own unique dynamics (Sukopp 1998). Urban environments represent the convergence of humans from around the world, any plant or animal species those humans might have brought with them, and infrastructures such as roads, sewers, and tall buildings. Despite the complexity that these ecosystems present, they are often overlooked by ecologists because more traditional ecology has focused on `natural' systems. In urban systems, however, human influence is a primary ecological factor (McDonnell and Niemelä 2011). In recent years, ecologists have begun studying the urban environment just as they would a natural environment, in order to understand the novel environmental conditions this setting presents to the organisms that live there (Wu 2014).

In this study, we focus on the urban ecology of trees in San Francisco, California. Urban trees can play a major role in shaping the ecosystem of a city. Just as the trees in a forest have a considerable impact on its climate and ecology, trees in cities can have notable effects on a city's environment. For example, plant life in cities can impact temperature, air quality, and other aspects of human health (Willis and Petrokofsky 2017). The urban heat island effect, which occurs when `islands' of heat form as heat gets trapped between tall buildings, is one of the most well-documented unique urban anthropogenic environmental conditions (Oke 1973). Trees in urban environments have been shown to interact with these city-specific environmental factors (Kong et al. 2014). For example, trees in cities have been shown to improve urban air quality by taking up significant amounts of carbon dioxide from city air (Nowak et al. 2014). As pollution generated in urban centers is one of the major contributing factors to worldwide pollution, trees in urban environments may have a role in managing the environmental impacts of urbanization (Alberti et al. 2003).

Due to their importance, there is growing interest in understanding and maintaining sustainable urban ecologies. One potentially major factor influencing health of plants both in nature and cities is the plant microbiome. As the widespread availability of DNA sequencing has made it possible to characterize microbial communities more easily and comprehensively, the microbiome has become a major area of interest in numerous organisms (Kyrpides, Eloe-Fadrosh, and Ivanova 2016). Just as the emerging field of human microbiome study has revealed that symbiotic, nonpathogenic microbes can have major impacts on human health (David et al. 2014), plant life is also host to numerous symbiotic microbes. Similar to the human microbiome, the plant microbiome contains great diversity and is comprised of multiple distinct communities in various tissue systems such as the roots (rhizosphere), leaf surface (phylosphere), and leaf interior (endosphere) (Turner, James, and Poole 2013). All of these microbiomes can have an impact on their host's physiology. For instance, bacterial root microbes have been shown to play a role in the growth of various plant species(Gaiero et al. 2013). They can also play a variety of roles in host physiology, depending on their location within the plant (Schlaeppi and Bulgarelli 2015). These communities can be quite dynamic, and vary with factors such as plant age (Cavaglieri, Orlando, and Etcheverry 2009). while each of these sets of interactions is important, here we focus on the microbial ecology of fungal microorganisms living in the endosphere.

The endosphere is is an ideal system for biogeographic studies, both because it is highly diverse and because it is a major interface between the host plant and its environment (Meyer and Leveau 2012).

Endophytic microbes are naturally found in the spacious interior of leaves, whether introduced by natural wounds or openings in the leaf surface, or through penetrating the plant surface with hydrolytic enzymes (Hallmann et al. 1997) Although some of these microbes may be latent pathogens or decomposers waiting for the leaf to die, others are mutualists that may confer a benefit to their host (Carroll 1988). In wild grasses, symbiotic fungi have been shown to protect their hosts by discouraging herbivory, and can even affect host reproductive viability in those same systems (Clay 1988). In controlled settings, inoculation experiments have shown that specific species of endophytes can have an impact on their host's overall health, including factors such as resistance and susceptibility to disease (Busby, Ridout, and Newcombe 2016). In nature, the fungal microbial communities can have impacts on their host's physiology, such as limiting pathogen damage (Arnold et al. 2003).

In the wild, endophytic communities display species diversity comparable to that of any macroscopic community, even among individual trees from the same species (Gazis, Rehner, and Chaverri, n.d.). However, what factors influence this diversity and to what extent is still poorly understood. The biodiversity of these communities can be quite high, especially in areas like tropical forests (Arnold and Lutzoni 2007). In such natural settings, plant-associated microbial community compositions can show clear biogeographic structure (Andrews and Harris 2000). The urban setting, however, may be distinct, because factors such as rainfall and elevation will likely be less apparent over a smaller geographic area, but new factors such as proximity to roads and tall buildings may introduce effects of their own. Studies of suburban forests in Japan have indicated that an urban setting has a notable impact on endophytic diversity (Matsumura and Fukuda 2013). However, the full impact of urban environmental factors on endophytic communities has yet to be completely understood.

In this study, used culturing and barcode gene sequencing to identify the species makeup of endophytic communities in the New Zeland Christmas Tree, \emph{Metrosideros excelsa} (Myrtaceae), throughout San Francisco, CA to quantify the diversity and biogeographic patterns of foliar fungal communities in this urban environment. Based on studies in natural systems, we expected to find high community diversity both within and between sites, as well as a degree of biogeographic structure to these community compositions. We also expected that urban environmental factors may play a role in shaping the biogeography in these endophytic communities.

\hypertarget{methods}{%
\section*{Methods}\label{methods}}
\addcontentsline{toc}{section}{Methods}

\hypertarget{host-selection}{%
\subsection*{Host Selection}\label{host-selection}}
\addcontentsline{toc}{subsection}{Host Selection}

We chose to focus on \emph{Metrosideros excelsa} individuals (Figure 1). \emph{Metrosideris excelsa} was an ideal species to choose for this study because it is widely planted throughout San Francisco, which indicates that it likely has a major impact on the urban environment, and that we would be able to obtain samples from a large number of trees in a variety of locations throughout the city. Although \emph{M. excelsa's} endophytic communities have been characterized in its native home of New Zeland, there have been few studies about these communities outside of its native environment or in an urban setting (McKenzie, Buchanan, and Johnston 1999). In a related Hawaiian species, \emph{Metrosideros polymorpha}, the species makeup of fungal endophyte communities has been shown to vary greatly with environmental factors such as elevation and rainfall (Zimmerman and Vitousek 2012).

\hypertarget{site-selection}{%
\subsection*{Site Selection}\label{site-selection}}
\addcontentsline{toc}{subsection}{Site Selection}

We selected six sites across the city to collect leaf samples from. When selecting sites, we took factors such as traffic levels, temperature, proximity to tall buildings, elevation, and proximity to the ocean into account. In selecting the sites, we wanted to capture the greatest range of potential urban climates as we could. We used the Urban Forest Map, which documents the location and species of every tree in San Francisco, to choose unique locations around the city with sufficient densities of Metrosideros excelsa individuals (Figure 1).

\hypertarget{sample-collection}{%
\subsection*{Sample Collection}\label{sample-collection}}
\addcontentsline{toc}{subsection}{Sample Collection}

We collected small branches from 5 trees in each of these sites using a clipper pole, collected at least 3 sun-facing outer branches from each tree. We controlled for leaf age by picking leaves that appeared to be older than one year. Because \emph{M. excelsa} is an evergreen tree and the newer leaves are likely to contain fewer fungi, we only collected branches that contained dark green leaves that appeared to be at least one year old. We collected all samples on the same day, August 26, 2017, to ensure that daily weather patterns and seasonal effects would not have an impact on the microbial community composition. Once collected, leaves were stored in labeled plastic bags and stored at 4\(^{\circ}\)C until culturing. All leaves were processed within 48 hours of collection.

\hypertarget{culturing-methods}{%
\subsection*{Culturing Methods}\label{culturing-methods}}
\addcontentsline{toc}{subsection}{Culturing Methods}

After we brought the branches back, we selected a subset of dark green asymptomatic leaves to culture. We surface-sterilized the leaves by first rinsing them with distilled water, then rinsing them in a petri dish with 95\% ethanol for 10 seconds, 10\% NaOCl for 2 minutes, then 70\% ethanol for another 2 minutes, emptying the dish between rinses and leaving it closed until inside a sterile biosafety cabinet after the last rinse. Then we cut the leaves into small (2mm x 2mm) pieces and put them into 1.5 mL microcentrifuge slant tubes filled with 2\% VWR-brand malt extract agar (MEA). We used MEA because it is considered the standard media for isolating the largest variety of fungal species. For each tree, we prepared 6 leaves and made 100 tubes, except for the trees from the downtown site. For these trees, we prepared 140 tubes per tree because they had low isolation frequencies in a preliminary sampling. All leaves were prepared this way within 48 hours of the initial leaf sampling, to prevent death of the leaf tissue from altering the fungal community composition.

After two weeks, we evaluated the tubes for fungal growth and subcultured the emergent fungi from tubes with growth onto 35mm 2\% MEA in order to better evaluate their morphotypes and accumulate sufficient tissue for future barcode gene sequencing and voucher preparation. We re-evaluated and subcultured these tubes in the following months to capture any late-growing fungi. Each pure culture was vouchered in sterile dI water and stored in a living culture collection in the Zimmerman Lab at the University of San Francisco.

\hypertarget{molecular-methods}{%
\subsection*{Molecular Methods}\label{molecular-methods}}
\addcontentsline{toc}{subsection}{Molecular Methods}

We extracted DNA from fungal mycelium using the Sigma RED Extract `n Amp DNA extraction kit and following previously published protocols (U'Ren et al. 2012). First, we added fungal tissue to sterile tubes filled with 1 mm zirconium oxide beads, then added 100 \(\mu\)L of Extract 'n' Amp DNA extraction solution. Next, we put the tubes in the bead-beater for one minute. The samples were then placed on a heat block at 95\(^{\circ}\)C for 10 minutes. After the heating step, we added a dilution buffer to each tube and stored them at 4\(^{\circ}\)C until PCR.

We performed PCR on the Internal Transcribed Spacer region, a commonly-accepted fungal barcode locus (Schoch et al. 2012), using the ITS1F forward primer (5'-CTT GGT CAT TTA GAG GAA GTA A-3') and ITS4 reverse primer (5'-TCC TCC GCT TAT TGA TAT GC-3'). For each PCR reaction, we used 1 \(\mu\)L of template DNA, 10 \(\mu\)L Extract 'n Amp Taq polymerase, 6.4 \(\mu\)L PCR-grade water, 1 \(\mu\)L bovine serum albumin, 0.8 \(\mu\)L ITS1F forward primer, and 0.8 \(\mu\)L ITS4 reverse primer (U'Ren et al. 2012). For the PCR reaction, we used a BioRAD T100 thermal cycler with the following cycles: 95\(^{\circ}\)C for 3 minutes, 95\(^{\circ}\)C for 30 seconds, 54\(^{\circ}\)C for 30 seconds, 72\(^{\circ}\)C for 30 seconds, repeat steps 2-4 34 times and then, 72\(^{\circ}\)C for 10 minutes. To ensure that the fungal DNA successfully amplified, and that the master mix was not contaminated, we ran 5 \(\mu\)L of each sample and a negative PCR control on a 1\% agarose gel with 1X Tris-acetate-EDTA (TEA) buffer and SYBR Safe. The gel was run at 120 volts for 20 minutes and visualized using UV transmission. Successful PCR samples with clean negative controls were kept at 4\(^{\circ}\)C until sequencing preparation.

To prepare successful samples for Sanger sequencing, they were first cleaned with 1 \(\mu\)L Thermo Fisher Shrimp Alkaline Phopsphatase Exonuclease (ExoSap) per sample. To clean the samples, we used the following cycle on the same BioRAD T100 thermal cycler: 37\(^{\circ}\)C for 15 minutes, 80\(^{\circ}\)C for 15 minutes, then infinite hold at 4\(^{\circ}\)C. After cleaning, samples were kept at 4\(^{\circ}\)C until they were ready to be sent for sequencing. Directly before being sent for sequencing, cleaned samples that showed bright bands on their gels were diluted with an additional 15 \(\mu\)L PCR water before sequencing, although a small number of samples that had faint bands on the gels were not diluted. Cleaned samples were sent to MCLabs (South San Francsisco, CA) for Sanger sequencing.

\hypertarget{computational-and-statistical-methods}{%
\subsection*{Computational and Statistical Methods}\label{computational-and-statistical-methods}}
\addcontentsline{toc}{subsection}{Computational and Statistical Methods}

We used Geneious 11.1 to manually clean and trim the Sanger sequencing data, and to identify and remove failed and low-quality sequences (Kearse et al. 2012). Sanger sequence quality was determined using Geneious's chromatogram, and sequences that appeared to have multiple strong signals at a given position were discarded because they likely came from a mixed culture. Usable sequences were cleaned by trimming the ends with low-quality reads, clarifying any ambiguity codes, and resolving dye blobs.

We used Mothur version 1.39.5 to determine Operational Taxonomic Units (OTUs), which are groups of sequences categorized together based on similarity (Schloss et al. 2009). Next, we used R to analyze the resulting OTU table. This included using the `vegan' package to run and plot a Non-Metric Multidimensional Scaling (NMDS) ordination, a non-parametric technique used to visualize high-dimensional community data in only two dimensions. We used vegan (Oksanen et al. 2018) to calculate species accumulation curves to determine species richness from the OTU data. We used the Tree-Based Alignment Selector (TBAS) toolkit to construct a phylogeny and assign taxonomies to the data (Carbone et al. 2017). This toolkit matches unknown ITS sequences to the most similar ITS sequences in a large multi-gene phylogeny of confidently-assigned taxa.

Significant differences in isolation frequency and tree diameter were determined using a nonparametric Kruskal--Wallis one-way test of variance in R 3.4.4 (R Core Team 2018). All figures and the final manuscript were generated using an R Markdown script, which is available here: \url{https://github.com/ZimmermanLab/SF-metrosideros-endophytes/blob/master/eg_thesis.Rmd}

\hypertarget{results}{%
\section*{Results}\label{results}}
\addcontentsline{toc}{section}{Results}

Overall, we found high diversity in these urban endophytic communities. While some communities showed a low number of fungal isolates, species richness analyses showed that in nearly all of the microbial communities analyzed, our sampling did not encompass the complete microbial diversity within these microbiomes. While the species diversity within some sites exceeds 20 distinct fungal OTUs, there appears to be greater species diversity in these endophytic communities than this study was able to document. Furthermore, the taxonomic variation between different sites is also considerable, as there is no taxon that dominates in all sites, although there are several taxa that show a degree of prominence in all sites.

\hypertarget{isolation-frequency}{%
\subsection*{Isolation Frequency}\label{isolation-frequency}}
\addcontentsline{toc}{subsection}{Isolation Frequency}

The isolation frequency, or the percentage of leaf pieces that yielded fungal isolates, varied significantly between sites (Kruskal Wallis p \textless{} 0.01, Figure 2 A.). In most sites, the isolation frequency also varies between trees, especially in the Bay site. Trees within the Bay site also show the greatest variation in diameter at breast height (DBH) (Figure 2 B.). The only site that does not show as consistent variation in isolation frequency is the Downtown site, which is also the site with the smallest isolation frequencies.

\hypertarget{diversity-patterns}{%
\subsection*{Diversity Patterns}\label{diversity-patterns}}
\addcontentsline{toc}{subsection}{Diversity Patterns}

There were 88 total OTUs found among the 30 different trees. Both isolation frequency and number of fungal species found varied notably between trees. Species accumulation curves showed that the total amount of fungal diversity has not been completely sampled in any of the sites (Figure 3).

The most abundant fungal class among all communities is Dothideomycetes, followed by Sordariomycetes, which each had over 300 sequence (Table 1). Dothidewas predominant in most sites except for the Downtown and Bay sites. In both of these sites, Sordariomycetes is the most common class instead. There are several classes that are either absent or present in small numbers in most sites, but more abundant in one or several sites. For example, Eurotiomycetes are more common in the Downtown and Freeway sites, and Leotiomycetes are only abundant in the Mt. Davidson site.

\hypertarget{biogeographic-patterns}{%
\subsection*{Biogeographic Patterns}\label{biogeographic-patterns}}
\addcontentsline{toc}{subsection}{Biogeographic Patterns}

The similarities between microbial communities of these trees vary (Figure 5). Some sites show very little compositional similarity, while others cluster more tightly, indicating a greater degree of within-site beta diversity. However, sites that were closer together geographically also tended to have greater microbial community similarities.

\hypertarget{discussion}{%
\section*{Discussion}\label{discussion}}
\addcontentsline{toc}{section}{Discussion}

In this study, we sought to characterize the foliar microbiome of \emph{Metrosideros excelsa} trees across the city of San Francisco. We found that he microbial composition of these urban trees' leaves varies in many aspects, from number of fungal isolates to the identities of said isolates. This variation could be explained by numerous environmental factors, as well as host physiological factors, such as the age and size of the tree. Each aspect of these complex microbiomes is likely influenced by several of these factors at once.

\hypertarget{isolation-frequency-and-tree-size}{%
\subsection*{Isolation Frequency and Tree size}\label{isolation-frequency-and-tree-size}}
\addcontentsline{toc}{subsection}{Isolation Frequency and Tree size}

Both isolation frequency and DBH varied greatly from one site to the other, and some sites show considerable within-site differences in these factors as well. The trees in the Bay show the greatest range in both isolation frequency and DBH (Figure 2 A and B), indicating that the size of a tree may have a correlation with the number of fungal endophytes found within its leaves. The Mt. Davidson trees have a larger median and range of DBH than the ocean or freeway trees (Figure 2 B.), and also have a higher median isolation frequency, which could indicate that trees with a larger DBH may also have a higher number of fungal endophytes.

Just as it shows the greatest range of isolation frequencies and tree sites, the Bay site also has some of the least similarity between its fungal communities on the NMDS ordination, and the two trees with vastly different communities are also the smallest (Figure 5). In this instance, it appears that the size of the trees has the largest impact on the endophytic communities. Furthermore, the trees in the Mt. Davidson site, which have some of the most similar communities (Figure 5), also have fairly large trees with similar DBH. The fact that these trees have similar sizes in addition to similar community composition indicates that a tree's size may have an impact on its endophytic communities.

Although there appears to be a general pattern with larger trees hosting a greater number of fungal endophytes, DBH cannot explain all of the variation in isolation frequencies, as demonstrated by the Downtown and Balboa sites. Although the trees from the downtown site have a higher median DBH (Figure 2 B.), the trees from the Balboa site have considerably more endophytes (Figure 2 A.), which suggests that larger trees do not necessarily have a higher number of fungal endophytes.

\hypertarget{diversity-patterns-1}{%
\subsection*{Diversity Patterns}\label{diversity-patterns-1}}
\addcontentsline{toc}{subsection}{Diversity Patterns}

Endophytic microbiomes in tree leaves are known to be highly diverse in natural systems, and that also appears to be the case in these urban trees. he species accumulation curves indicate that even in the most well-represented sites, these are still many potentially undiscovered OTUs within these endophytic communities (Figure3). Additionally, there is a considerable amount of taxonomic variation between certain sites, in addition to the generally high diversity of each site (Figure 4). In such cases, it is likely that environmental factors play a role in shaping the endophytic communities of these trees. While the impact of environmental factors may be less evident when considering the number of fungal endophytes in a tree's microbiome, it becomes more apparent when looking at the identities of these endophytes. The composition of these communities can vary greatly among trees with similar isolation frequencies, as demonstrated by the taxonomic composition of the Mt. Davidson and Bay sites (Figure 4). In general, there was higher compostitional similarity between trees from the same location than between trees of a similar size (Figure 5).

\hypertarget{biogeographic-patterns-1}{%
\subsection*{Biogeographic Patterns}\label{biogeographic-patterns-1}}
\addcontentsline{toc}{subsection}{Biogeographic Patterns}

There appears to be a degree of biogeographic structure to the endophytic communities within these urban trees. Communities of trees from the same site generally cluster closer together than trees from different sites, although some sites have much greater within-site diversity than others (Figure 5). It may be more likely that some of the sites that show great within-site diversity simply appear that way due to undersampling. Although the Balboa site appears to have a high isolation frequency (Figure 2 A), many of the slant tubes that showed fungal growth failed to grow into a larger culture, leading to a low number of usable sequences (Figure 3). This low number of sequences per tree may indicate that the divergence in community composition that appears to be present might be due to the low sample size, rather than actual difference between the microbial communities between the Balboa trees (Figure 5).

However, even sites with low within-site diversity appear to be more similar to other trees from geographically close sites. For example, the Downtown trees do not cluster with the Mt. Davidson, Balboa, or Ocean trees in the ordination, indicating that they have fairly different fungal communities (Figure 5). One notable exception to this pastern is the Downtown and Freeway sites, which cluster fairly close together (Figure 5) despite being distant geographically (Figure 1). In this case, it is likely that there are environmental factors that cause the communities to be more compositionally similar. One of the most likely environmental similarities between these two sites is that they are both exposed to high traffic levels, with one site located in a bustling downtown and the other located near two large freeways. It is possible that carbon dioxide emissions, particulates, or both of these pollutants could be having an effect on both these trees and their microbiomes. Endophytic bacteria has been shown to have a phytoremediating effect (Afzal, Khan, and Sessitsch 2014), so it is possible that endophytic fungi may be playing a similar role in helping these plants adapt to traffic-heavy conditions.

\hypertarget{community-composition-compared-to-related-trees-in-other-locations}{%
\subsection*{Community composition compared to related trees in other locations}\label{community-composition-compared-to-related-trees-in-other-locations}}
\addcontentsline{toc}{subsection}{Community composition compared to related trees in other locations}

In widely-distributed plant species, fungal endophyte composition can differ quite considerably in a species from its native home to the places it has been introduced (Taylor, Hyde, and Jones 1999). \emph{M. excelsa} is native to New Zeland, but it and its close relatives in the \emph{Myrtaceae} family are widespread across the world. Across all locations, the most abundant classes found within these \emph{M. excelsa} fungal communities were Dothideomycetes and Sordariomycetes by a notable margin (Table 1). Interestingly, at broad taxonomic levels such as Class, the endophytes found in this study appear to be similar to those found in other Myrtaceae trees elsewhere. For instance, a study of \emph{Myrtaceae} trees in South America, the first and second most common fungal isolates were also identified as Dothideomycetes followed by Sordariomycetes (Vaz et al. 2014). Dothideomycetes and Sordariomycetes are also the first and second most abundant classes in the endophytic communities of \emph{M. excelsa's} close relative in Hawaii, \emph{Metrosideros polymorpha}, as well (Zimmerman and Vitousek 2012). This could indicate that these fungal classes are well-adapted to living as endophytes within these trees, but it is also possible that these communities are actually quite different when considered at a finer taxonomic resolution.

\hypertarget{comparison-of-trees-in-san-francisco-and-new-zealand}{%
\subsection*{Comparison of trees in San Francisco and New Zealand}\label{comparison-of-trees-in-san-francisco-and-new-zealand}}
\addcontentsline{toc}{subsection}{Comparison of trees in San Francisco and New Zealand}

In order to compare the foliar microbial populations found in M. excelsa throughout San Francisco to its microbial populations in its native land of New Zealand, we both used genbank and performed a literature search. First, we searched Google Scholar for papers mentioning M. excelsa fungal endophytes and then, more generally, any foliar fungus inhabiting M. excelsa leaves. After this search, we looked through the papers cited by those we originally found in order to find additional recorded fungal isolates.
Then, we searched GenBank for fungal ITS sequences that had M. excelsa listed as their host organism. We performed this search using the parameters host=``Metrosideros excelsa,'' organism=fungi, in order to limit our results to fungal sequences. This search yielded several new sequences that weren't included in the previous search because they were unpublished. Using both of these methods, we were able to find about 30 taxa to include as a point of comparison, although many of these were pathogenic instead of endophytic.

When comparing \emph{M. excelsa's} endophytic populations found in the literature to those that we collected, we found that the taxa between the regions were somewhat similar on an order level, and even moreso on a class level. Only one of the orders that was extremely common in New Zealand (Mycospharales) was absent in San Francisco, and only two of the most common orders in San Francisco (Capnodiales and Pezizales) were absent in New Zealand. However, most of the families that are found in New Zealand are absent in San Francisco, and vice versa. We only found four families in common: Mycosphaerellaceae, Glomerellaceae, Nectriaceae, and Helotiaceae. Of these, Mycosphaerellaceae is the only family that is present in New Zealand very prevalent (over 50 isolates) in San Francisco. The other families that they had in common were only present in very small numbers in San Francisco, and we were unable to find documentation of three most prevalent families in San Francisco (Xylariaceae, Botryosphaeriaceae, and Cladosporiaceae) being isolated from leaves in New Zealand. Although this could be due to the small sample size, it suggests that \emph{M. excalsa's} endophytic populations may be influenced by their environment, rather than being carried over from their native home. However, further research into both the endophytic populations of other trees in San Francisco and M. excelsa in its native range would be required to draw further conclusions.

Additionally, it is important to note the difference in methods between how we collected our samples and how others studying \emph{M. excelsa} in New Zealand collected theirs. Whereas we took care to only collect samples from unfallen, asymptomatic leaves, most of the papers referenced in this comparison use either leaf litter, dead leaves, or obviously symptomatic leaves. Therefore, it is reasonable to infer that the data we are comparing our results to is biased towards pathogenic and decomposing fungi, and underrepresents the non-harmful endophytic populations in M. excelsa's native range. Additionally, the number of fungal isolates we were able to find using literature and genbank searches was much lower than the number of samples we collected ourselves, making quantitative comparison unfeasible. However, we think that such comparisons are still worthwhile as a preliminary look into the extent that \emph{M. excelsa} in San Francisco may maintain its native endophytic populations.

\hypertarget{conclusions}{%
\section*{Conclusions}\label{conclusions}}
\addcontentsline{toc}{section}{Conclusions}

These findings indicate that the endophytic microbiomes of urban trees are complex and diverse, and may show a degree of biogeographic structure that reflects their natural counterparts. Additionally, it is likely that urban environmental factors play a considerable role in shaping the endophytic communities of these trees. This study has demonstrated that the urban endophytic microbiomeis highly diverse and may be structured by unique urban environmental factors. Nearly all of the species accumulation curves indicate that the full diversity of these endophytic communities has yet to be sampled (Figure 3). Even in the small geographic area of San Francisco, we found notable trends in microbiome composition that appear to vary with uniquely urban environmental factors, such as traffic. A combination of environmental factors and host physiology therefore appear to be the driving force behind the diversity of these microbiomes. While it is difficult to determine, given the data we have, the exact mechanisms that influence these communities, the amount of species diversity and biogeographic structure indicate that the foliar microbiomes of urban trees may be just as complex and dynamic as those of trees in nature, and warrant further study.

\hypertarget{acknowledgements}{%
\section*{Acknowledgements}\label{acknowledgements}}
\addcontentsline{toc}{section}{Acknowledgements}

\begin{table}[!h]

\caption{\label{tab:abund}Abundances of classes of Ascomycota across all trees and sites.}
\centering
\begin{tabular}[t]{lr}
\toprule
Fungal class & Number of sequences\\
\midrule
Dothideomycetes & 450\\
Sordariomycetes & 341\\
Eurotiomycetes & 57\\
Pezizomycetes & 52\\
Leotiomycetes & 19\\
\bottomrule
\end{tabular}
\end{table}

\begin{figure}
\centering
\includegraphics{gibson2021_files/figure-latex/map-1.pdf}
\caption{\label{fig:map}A map of the locations sampled across the city of San Francisco, California, USA. Five trees from each site were sampled. Sites were selected to span the range of environmental conditions across the city.}
\end{figure}

\begin{figure}
\centering
\includegraphics{gibson2021_files/figure-latex/isolation-1.pdf}
\caption{\label{fig:isolation}Isolation frequencies (A) and tree diameters (B) at each site. Isolation frequency (A) is a measure of how many slant tubes showed signs of fungal growth, out of how many total slant tubes were made. 100 slant tubes were made for each tree except for the trees in the downtown site, which had 140 slant tubes per tree because they had low isolation frequencies during the initial sampling.}
\end{figure}

\begin{figure}
\centering
\includegraphics{gibson2021_files/figure-latex/rarefaction-1.pdf}
\caption{\label{fig:rarefaction}Species accumulation curve showing species richness in each site. Each line represents the combined species richness of all trees in one site.}
\end{figure}

\begin{figure}
\centering
\includegraphics{gibson2021_files/figure-latex/bar-graph-1.pdf}
\caption{\label{fig:bar-graph}Normalized relative abundances of Ascomycota taxa in each site.}
\end{figure}

\begin{figure}
\centering
\includegraphics{gibson2021_files/figure-latex/ordination-1.pdf}
\caption{\label{fig:ordination}NMDS ordination of community compositions. Each point represents the endophytic community of one tree, and the size of the point corresponds to that tree's DBH, while the color of said point corresponds to the site that tree is from. Points that are closer together indicate that the trees they represent have similar community compositions. The ellipses show the standard error around the centroid of all points within a site, and are also color-coded according to which site they represent.}
\end{figure}

\hypertarget{references}{%
\section*{References}\label{references}}
\addcontentsline{toc}{section}{References}

\hypertarget{refs}{}
\begin{CSLReferences}{1}{0}
\leavevmode\vadjust pre{\hypertarget{ref-afzal2014endophytic}{}}%
Afzal, Muhammad, Qaiser M Khan, and Angela Sessitsch. 2014. {``Endophytic Bacteria: Prospects and Applications for the Phytoremediation of Organic Pollutants.''} \emph{Chemosphere} 117: 232--42.

\leavevmode\vadjust pre{\hypertarget{ref-alberti2003integrating}{}}%
Alberti, Marina, John M Marzluff, Eric Shulenberger, Gordon Bradley, Clare Ryan, and Craig Zumbrunnen. 2003. {``Integrating Humans into Ecology: Opportunities and Challenges for Studying Urban Ecosystems.''} \emph{AIBS Bulletin} 53 (12): 1169--79.

\leavevmode\vadjust pre{\hypertarget{ref-andrews2000ecology}{}}%
Andrews, John H, and Robin F Harris. 2000. {``The Ecology and Biogeography of Microorganisms on Plant Surfaces.''} \emph{Annual Review of Phytopathology} 38 (1): 145--80.

\leavevmode\vadjust pre{\hypertarget{ref-arnold2007diversity}{}}%
Arnold, A Elizabeth, and Francois Lutzoni. 2007. {``Diversity and Host Range of Foliar Fungal Endophytes: Are Tropical Leaves Biodiversity Hotspots?''} \emph{Ecology} 88 (3): 541--49.

\leavevmode\vadjust pre{\hypertarget{ref-arnold2003fungal}{}}%
Arnold, A Elizabeth, Luis Carlos Mejia, Damond Kyllo, Enith I Rojas, Zuleyka Maynard, Nancy Robbins, and Edward Allen Herre. 2003. {``Fungal Endophytes Limit Pathogen Damage in a Tropical Tree.''} \emph{Proceedings of the National Academy of Sciences} 100 (26): 15649--54.

\leavevmode\vadjust pre{\hypertarget{ref-Busby2016}{}}%
Busby, Posy E., Mary Ridout, and George Newcombe. 2016. {``Fungal Endophytes: Modifiers of Plant Disease.''} \emph{Plant Molecular Biology} 90 (6): 645--55. \url{https://doi.org/10.1007/s11103-015-0412-0}.

\leavevmode\vadjust pre{\hypertarget{ref-tbas}{}}%
Carbone, Ignazio, James B. White, Jolanta Miadlikowska, A. Elizabeth Arnold, Mark A. Miller, Frank Kauff, Jana M. U'Ren, Georgiana May, and François Lutzoni. 2017. {``T-BAS: Tree-Based Alignment Selector Toolkit for Phylogenetic-Based Placement, Alignment Downloads and Metadata Visualization: An Example with the Pezizomycotina Tree of Life.''} \emph{Bioinformatics} 33 (8): 1160--68. \url{https://doi.org/10.1093/bioinformatics/btw808}.

\leavevmode\vadjust pre{\hypertarget{ref-carroll1988fungal}{}}%
Carroll, George. 1988. {``Fungal Endophytes in Stems and Leaves: From Latent Pathogen to Mutualistic Symbiont.''} \emph{Ecology} 69 (1): 2--9.

\leavevmode\vadjust pre{\hypertarget{ref-cavaglieri2009rhizosphere}{}}%
Cavaglieri, Lilia, Julieta Orlando, and Miriam Etcheverry. 2009. {``Rhizosphere Microbial Community Structure at Different Maize Plant Growth Stages and Root Locations.''} \emph{Microbiological Research} 164 (4): 391--99.

\leavevmode\vadjust pre{\hypertarget{ref-clay1988fungal}{}}%
Clay, Keith. 1988. {``Fungal Endophytes of Grasses: A Defensive Mutualism Between Plants and Fungi.''} \emph{Ecology} 69 (1): 10--16.

\leavevmode\vadjust pre{\hypertarget{ref-david2014diet}{}}%
David, Lawrence A, Corinne F Maurice, Rachel N Carmody, David B Gootenberg, Julie E Button, Benjamin E Wolfe, Alisha V Ling, et al. 2014. {``Diet Rapidly and Reproducibly Alters the Human Gut Microbiome.''} \emph{Nature} 505 (7484): 559.

\leavevmode\vadjust pre{\hypertarget{ref-gaiero2013inside}{}}%
Gaiero, Jonathan R, Crystal A McCall, Karen A Thompson, Nicola J Day, Anna S Best, and Kari E Dunfield. 2013. {``Inside the Root Microbiome: Bacterial Root Endophytes and Plant Growth Promotion.''} \emph{American Journal of Botany} 100 (9): 1738--50.

\leavevmode\vadjust pre{\hypertarget{ref-gazis2011}{}}%
Gazis, Romina, Stephen Rehner, and Priscila Chaverri. n.d. {``Species Delimitation in Fungal Endophyte Diversity Studies and Its Implications in Ecological and Biogeographic Inferences.''} \emph{Molecular Ecology} 20 (14): 3001--13. \url{https://doi.org/10.1111/j.1365-294X.2011.05110.x}.

\leavevmode\vadjust pre{\hypertarget{ref-hallmann1997bacterial}{}}%
Hallmann, J, A Quadt-Hallmann, WF Mahaffee, and JW Kloepper. 1997. {``Bacterial Endophytes in Agricultural Crops.''} \emph{Canadian Journal of Microbiology} 43 (10): 895--914.

\leavevmode\vadjust pre{\hypertarget{ref-geneious}{}}%
Kearse, Matthew, Richard Moir, Amy Wilson, Steven Stones-Havas, Matthew Cheung, Shane Sturrock, Simon Buxton, et al. 2012. {``Geneious Basic: An Integrated and Extendable Desktop Software Platform for the Organization and Analysis of Sequence Data.''} \emph{Bioinformatics} 28 (12): 1647--49. \url{https://doi.org/10.1093/bioinformatics/bts199}.

\leavevmode\vadjust pre{\hypertarget{ref-salford40381}{}}%
Kong, F, H Yin, P James, LR Hutyra, and HS He. 2014. {``Effects of Spatial Pattern of Greenspace on Urban Cooling in a Large Metropolitan Area of Eastern China.''} \emph{Landscape and Urban Planning} 128: 35--47. \url{https://doi.org/10.1016/j.landurbplan.2014.04.018}.

\leavevmode\vadjust pre{\hypertarget{ref-kyrpides2016microbiome}{}}%
Kyrpides, Nikos C, Emiley A Eloe-Fadrosh, and Natalia N Ivanova. 2016. {``Microbiome Data Science: Understanding Our Microbial Planet.''} \emph{Trends in Microbiology} 24 (6): 425--27.

\leavevmode\vadjust pre{\hypertarget{ref-MATSUMURA2013191}{}}%
Matsumura, Emi, and Kenji Fukuda. 2013. {``A Comparison of Fungal Endophytic Community Diversity in Tree Leaves of Rural and Urban Temperate Forests of Kanto District, Eastern Japan.''} \emph{Fungal Biology} 117 (3): 191--201. https://doi.org/\url{https://doi.org/10.1016/j.funbio.2013.01.007}.

\leavevmode\vadjust pre{\hypertarget{ref-mcdonnell2011history}{}}%
McDonnell, Mark J, and J Niemelä. 2011. {``The History of Urban Ecology.''} \emph{Urban Ecology}, 9.

\leavevmode\vadjust pre{\hypertarget{ref-mckenzie1999}{}}%
McKenzie, E. H. C., P. K. Buchanan, and P. R. Johnston. 1999. {``Fungi on Pohutukawa and Other Metrosideros Species in New Zealand.''} \emph{New Zealand Journal of Botany} 37 (2): 335--54. \url{https://doi.org/10.1080/0028825X.1999.9512637}.

\leavevmode\vadjust pre{\hypertarget{ref-meyer2012microbiology}{}}%
Meyer, Katrin M, and Johan HJ Leveau. 2012. {``Microbiology of the Phyllosphere: A Playground for Testing Ecological Concepts.''} \emph{Oecologia} 168 (3): 621--29.

\leavevmode\vadjust pre{\hypertarget{ref-NOWAK2014119}{}}%
Nowak, David J., Satoshi Hirabayashi, Allison Bodine, and Eric Greenfield. 2014. {``Tree and Forest Effects on Air Quality and Human Health in the United States.''} \emph{Environmental Pollution} 193: 119--29. https://doi.org/\url{https://doi.org/10.1016/j.envpol.2014.05.028}.

\leavevmode\vadjust pre{\hypertarget{ref-oke1973city}{}}%
Oke, Tim R. 1973. {``City Size and the Urban Heat Island.''} \emph{Atmospheric Environment (1967)} 7 (8): 769--79.

\leavevmode\vadjust pre{\hypertarget{ref-vegan}{}}%
Oksanen, Jari, F. Guillaume Blanchet, Michael Friendly, Roeland Kindt, Pierre Legendre, Dan McGlinn, Peter R. Minchin, et al. 2018. \emph{Vegan: Community Ecology Package}. \url{https://CRAN.R-project.org/package=vegan}.

\leavevmode\vadjust pre{\hypertarget{ref-r-citation}{}}%
R Core Team. 2018. \emph{R: A Language and Environment for Statistical Computing}. Vienna, Austria: R Foundation for Statistical Computing. \url{https://www.R-project.org/}.

\leavevmode\vadjust pre{\hypertarget{ref-schlaeppi2015plant}{}}%
Schlaeppi, Klaus, and Davide Bulgarelli. 2015. {``The Plant Microbiome at Work.''} \emph{Molecular Plant-Microbe Interactions} 28 (3): 212--17.

\leavevmode\vadjust pre{\hypertarget{ref-schloss2009introducing}{}}%
Schloss, Patrick D, Sarah L Westcott, Thomas Ryabin, Justine R Hall, Martin Hartmann, Emily B Hollister, Ryan A Lesniewski, et al. 2009. {``Introducing Mothur: Open-Source, Platform-Independent, Community-Supported Software for Describing and Comparing Microbial Communities.''} \emph{Applied and Environmental Microbiology} 75 (23): 7537--41.

\leavevmode\vadjust pre{\hypertarget{ref-schneider2009new}{}}%
Schneider, Annemarie, Mark A Friedl, and David Potere. 2009. {``A New Map of Global Urban Extent from MODIS Satellite Data.''} \emph{Environmental Research Letters} 4 (4): 044003.

\leavevmode\vadjust pre{\hypertarget{ref-schoch2012nuclear}{}}%
Schoch, Conrad L, Keith A Seifert, Sabine Huhndorf, Vincent Robert, John L Spouge, C Andre Levesque, Wen Chen, et al. 2012. {``Nuclear Ribosomal Internal Transcribed Spacer (ITS) Region as a Universal DNA Barcode Marker for Fungi.''} \emph{Proceedings of the National Academy of Sciences} 109 (16): 6241--46.

\leavevmode\vadjust pre{\hypertarget{ref-sukopp1998urban}{}}%
Sukopp, Herbert. 1998. {``Urban Ecology---Scientific and Practical Aspects.''} In \emph{Urban Ecology}, 3--16. Springer.

\leavevmode\vadjust pre{\hypertarget{ref-taylor_hyde_jones_1999}{}}%
Taylor, J. E., K. D. Hyde, and E. B. G. Jones. 1999. {``Endophytic Fungi Associated with the Temperate Palm, Trachycarpus Fortunei, Within and Outside Its Natural Geographic Range.''} \emph{New Phytologist} 142 (2): 335--46.

\leavevmode\vadjust pre{\hypertarget{ref-turner2013plant}{}}%
Turner, Thomas R, Euan K James, and Philip S Poole. 2013. {``The Plant Microbiome.''} \emph{Genome Biology} 14 (6): 209.

\leavevmode\vadjust pre{\hypertarget{ref-uren2012a}{}}%
U'Ren, Jana M, F Lutzoni, J Miadlikowska, A D Laetsch, and A Elizabeth Arnold. 2012. {``Host and Geographic Structure of Endophytic and Endolichenic Fungi at a Continental Scale.''} \emph{American Journal of Botany} 99 (5): 898--914.

\leavevmode\vadjust pre{\hypertarget{ref-vaz2014fungal}{}}%
Vaz, Aline BM, Sonia Fontenla, Fernando S Rocha, Luciana R Brandao, Mariana LA Vieira, Virginia De Garcia, Aristoteles Goes-Neto, and Carlos A Rosa. 2014. {``Fungal Endophyte \(\beta\)-Diversity Associated with Myrtaceae Species in an Andean Patagonian Forest (Argentina) and an Atlantic Forest (Brazil).''} \emph{Fungal Ecology} 8: 28--36.

\leavevmode\vadjust pre{\hypertarget{ref-Willis374}{}}%
Willis, Katherine J., and Gillian Petrokofsky. 2017. {``The Natural Capital of City Trees.''} \emph{Science} 356 (6336): 374--76. \url{https://doi.org/10.1126/science.aam9724}.

\leavevmode\vadjust pre{\hypertarget{ref-wu2014urban}{}}%
Wu, Jianguo. 2014. {``Urban Ecology and Sustainability: The State-of-the-Science and Future Directions.''} \emph{Landscape and Urban Planning} 125: 209--21.

\leavevmode\vadjust pre{\hypertarget{ref-Zimmerman13022}{}}%
Zimmerman, Naupaka B., and Peter M. Vitousek. 2012. {``Fungal Endophyte Communities Reflect Environmental Structuring Across a Hawaiian Landscape.''} \emph{Proceedings of the National Academy of Sciences} 109 (32): 13022--27. \url{https://doi.org/10.1073/pnas.1209872109}.

\end{CSLReferences}



\end{document}
